\chapter{Reconfiguration of satisfiability problems} \label{chap:SAT}

In this chapter, boolean satisifiability reconfiguration problems are presented. For decades the Boolean satisfiability problem also known
as SAT has fascinated scientific world. The craze behind it led to the celebrated theorem of Cook Levin, that SAT is $\NP$-complete.
Schaefer proposed in \cite{schaefer_complexity_1978} a framework for expressing variants of the satisfiability problem,
and showed a dichotomy theorem: the satisfiability problem for certain classes of Boolean formulas is in $\P$ while it is $\NP$-complete for
all other classes in the framework. In a single stroke, this result pinpoints the computational complexity of all well-known variants of SAT,
such as $3$-SAT, HORN $3$-SAT, NOT-ALL-EQUAL $3$-SAT, and 1-IN-$3$ SAT.

Since then, dichotomies or trichotomies have been established for several aspects of the satisfiability problem such as optimization [6,8,24],
counting [7], inverse satisfiability [23], minimal satisfiability [28], unique satisfiability [19], 3-valued satisfiability [3] and
propositional abduction [9]. Very recently, Gopalan et al. studied in [17,18] connectivity properties of the solution-space of Boolean
formulas, and investigated complexity issues on connectivity problems in Schaefer’s framework [31].

For Boolean satisfiability problems, the structure of the solution space is characterized by the solution graph, where the vertices are
the solutions, and two solutions are connected iff they differ in exactly one variable. In 2006, Gopalan et al. studied connectivity properties
of the solution graph motivated mainly by research on satisfiability algorithms and the satisfiability threshold. The connectivity problem (Conn) is
to decide whether the solutions of a given Boolean formula $\varphi$ on $n$ variables induce a connected subgraph of the $n$-dimensional hypercube,
while the st-connectivity problem (st-Conn) is to decide whether two specific solutions $s$ and $t$ of $\Phi$ are connected. They
proved dichotomies for the diameter of connected components and for the complexity of the st-connectivity question, and conjectured a trichotomy
for the connectivity question. Recently, the trichotomy was established in  \cite{schwerdtfeger2013computational}.

A direct application of st-connectivity in solution graphs are reconfiguration problems. Gopalan et al.’s results have also been applied
directly to reconfiguration problems, that arise when a step-by-step transformation
between two feasible solutions of a problem is searched, such that all intermediate results are feasible. The solutions
(satisfying assignments) of a formula $\Phi$ over n variables induce  a subgraph $G(\Phi)$ of the n-dimensional hypercube graph, that is, the
vertices are the solutions of $\Phi$, and two solutions are connected iff they differ in exactly one variable.

In the reconfiguration context the following general set-up is considered :
Given a Boolean formula $\varphi$ with $n$ Boolean variable and two satisfying assignments $s_{1}$ and $s_{2}$, is it possible to
transform $s_{1}$ to $s_{2}$ such that at each step, only one variable $x_i$ can be flipped and each intermediate assignment remains
feasible. \todo{Needs rework }


- Roadmap.


\section{Schaefer's framework}

\subsection{Basic concepts}
A CNF-formula is a Boolean formula of the form $C_{1} \land \dots \land C_{n}$, where each $C_i$ is a clause, that is, a finite disjunction
of literals. A $k$-CNF formula $(k \geq 1)$ is a CNF-formula where each $C_i$ has at most $k$ literals.

A \textit{logical relation} $R$ is a non-empty subset of $\{0,1\}^k$, for some $k \geq 1;$ $k$ is the \textit{arity} of $R$. A logical relation
is a function that takes as input a Boolean vector and returns a Boolean. For a set $\mathcal{S}$ of logical relations, a $\mathcal{S}$-formula
is a conjuntion of logical relations from $\mathcal{S}$ , where the arguments of each relation are freely chosen among a set of variables.



Let $\mathcal{S}$ be a finite set of logical relations. A CNF($\mathcal{S}$)\textit{-formula} over a set of variables $V = \{x_{1}, \dots, x_{n}\}$
is a finite conjunction $C_{1} \land \dots \land C_{n}$ of clauses built using relations from $\mathcal{S}$, variables from $V$, and the constants $0$ and $1$;
this means that each $C_{i}$ is an expression of the form $R()$, where $R \in \mathcal{S}$ is a relation of arity $k$, and each $C_{i}$ is a
variable in $V$ or one of the constants $0,1$. A $\textit{solution}$ if a CNF($\mathcal{S}$)-formula $\varphi$ is an assignment
$s = (a_{1}, \dots, a_{n})$ of Boolean values to the variables that makes every clause of $\varphi$ true. A CNF($\mathcal{S}$)-formula is
$\textit{satisfiable}$ if it has at least one solution.

\todo{To clean up}

\subsection{Statement of Results}
The $\textit{satisfiability problems}$ SAT($\mathcal{S}$) associated with a finite set $\mathcal{S}$ of logical relation asks :
Given a CNF($\mathcal{S}$) $\varphi$, is it satisfiable ?

\begin{theorem}Let $\mathcal{S}$ be a finite set of logical relations. If $\mathcal{S}$ is Schaefer, then SAT($\mathcal{S}$) is in
$\P$;  otherwise SAT($\mathcal{S}$) is $\NP-$complete.
\end{theorem}


\subsubsection{Dichotomie Results}

\section{Connectivity of CNF-Formulas / Gopalan et al's.}
Research has focused on the structure of the solution space only quite recently: One of the earliest studies on solution-space connectivity
was done for CNF($\mathcal{S}$)-formulas. In this paper, we are interested in the connectivity properties of the space of solutions
of CNF($\mathcal{S}$)-formulas. If $\varphi$ is a CNF($\mathcal{S}$)-formula with $n$ variables, then the solution graph
G($\varphi$) of $\varphi$ denotes the subgraph of the $n$-dimensional hypercube induced by the solutions of
$\varphi$. This means that the vertices of G($\varphi$) are the solutions of $\varphi$, and there is an edge between
two solutions of G($\varphi$) precisely when they differ in exactly one variable.

The following decision problems were considered in this context :
Gopalan et al. studied the following two decision problems for CNF($\mathcal{S}$)-formulas:

\begin{flushleft}
    The \textit{st-Connectivity Problem} ST-CONN($\mathcal{S}$) \\
    \textbf{Instance: } A CNF($\mathcal{S}$)-formula $\varphi$ and two satisfying assignments $s$ and $t$ of $\Phi$. \\
    \textbf{Question: } Is there a path from $s$ to $t$ in G($\varphi$) ? \\
\end{flushleft}

\begin{flushleft}
    The \textit{Connectivity Problem} CONN($\mathcal{S}$) \\
    \textbf{Instance: } A CNF($\mathcal{S}$)-formula $\Phi$ . \\
    \textbf{Question: } Is G($\Phi$) connected ? \\
\end{flushleft}




\section{Planar NAE 3-SAT Reconfiguration / Cardinal}