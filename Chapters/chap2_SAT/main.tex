\chapter{Reconfiguration of satisfiability problems} \label{chap:SAT}

In this chapter, boolean satisifiability reconfiguration problems are presented. For decades the Boolean satisfiability problem also known
as SAT has fascinated scientific world. In 1978 Schaefer proposed a framework for expressing variants of the satisfiability problem,
and showed a dichotomy theorem: the satisfiability problem for certain classes of Boolean formulas is in $\P$ while it is $\NP$-complete for
all other classes in the framework \cite{schaefer_complexity_1978}. In a single stroke, this result pinpoints the computational complexity
of all well-known variants of SAT, such as $3$-SAT, HORN $3$-SAT, NOT-ALL-EQUAL $3$-SAT, and 1-IN-$3$ SAT \cite{DBLP:journals/siamcomp/GopalanKMP09}.
Since then, dichotomies or trichotomies have been established for several aspects of the satisfiability problem such as optimization
\cite{CREIGNOU1995511,khanna_approximability_2001}, inverse satisfiability \cite{kavvadias_inverse_nodate}, minimal satisfiability \cite{KIROUSIS200320},
and 3-valued satisfiability \cite{10.1145/1120582.1120584}.


Continuing Schaefer's work, Gopalan et al. were interested by the connectivity properties of the solution graph motivated mainly by research
on satisfiability algorithms and the satisfiability threshold. The structure of the solution space of Boolean formulas can be characterized by
a $\textit{solution graph}$ where the vertices are the solutions (satisfying assignments) and two solutions are connection iff they differ in exactly
one variable. The connectivity problem (Conn) is to decide whether the solutions of a given Boolean formula $\varphi$ on $n$ variables induce a
connected subgraph of the $n$-dimensional hypercube, while the st-connectivity problem (st-Conn) is to decide whether two specific
solutions $s$ and $t$ of $\varphi$ are connected. Concerning the st-connectivity question, they proved dichtomies for the diameter of connected components
and the complexity. For the connectivity question, they conjectured a trichotomy. Building on this work, recently the trichotomy was
established by Schwerdtfeger in \cite{schwerdtfeger_computational_2015} and showed that Conn is either in $\P$, $\coNP$-complete, or $\PSPACE$-complete.


Recently, Gopalan et al.'s results have been applied to reconfiguration problems, mostly the st-connectivity and connectivity questions have been
applied to reconfiguration problems such as INDEPENDENT-SET RECONFIGURATION, GRAPH-k-COLORING RECONFIGURATION, and many complexity results were obtained.


\textit{Roadmap.} In section \ref{sec:schaefer's framework} we introduce Schaefer's framework and remarkable results, followed by section
\ref{sec:gopalan-et-al.'s-results} presenting Gopalan et al.'s results. Finally in section \ref{sec:a-computational-trichotomy} we present
Schwerdtfeger's results which establishes the trichotomy conjectured by Gopalan et al.'s for the connectivity problem.


% ----------------------------------------------------------------------------------------------------------------------------------
% ~~~~~~~~~~~~~~~~~~~~~~~~~~~~~~~~~~~~~~~~~~~~~~~~~~~~~~~~~~~~~~~~~~~~~~~~~~~~~~~~~~~~~~~~~~~~~~~~~~~~~~~~~~~~~~~~~~~~~~~~~~~~~~~~~~
% ----------------------------------------------------------------------------------------------------------------------------------
\section{Schaefer's framework}\label{sec:schaefer's framework}
Schaefer identified the complexity of every satisfiability problem SAT($\mathcal{S}$), where $\mathcal{S}$ ranges over all finite sets of
logical relations \cite{schwerdtfeger_computational_2015}. To state Schaefer’s main result, we need to define some basic concepts.
We will use the standard notions and definitions defined in \cite{schaefer_complexity_1978} and \cite{DBLP:journals/siamcomp/GopalanKMP09}.

\subsection{Preliminaries}

A CNF-formula is a Boolean formula of the form $C_{1} \land \dots \land C_{m}$, where each $C_i$ is a finite disjunction of literals where $C_i$
is referred to as a clause. A $k$-CNF formula $(k \geq 1)$ is a CNF-formula where each $C_i$ has at most $k$ literals.


A \textit{logical relation} $R$ is a non-empty subset of $\{0,1\}^n$, for some $n \geq 1$ where $n$ is the \textit{arity} of $R$.
A logical relation is a function that takes as input a Boolean vector and returns a Boolean. For a set $\mathcal{S}$ of logical relations, a
$\mathcal{S}$-formula is a conjuntion of logical relations from $\mathcal{S}$, where the arguments of each relation are freely chosen among
a set of variables.

For a finite set of relations $\mathcal{S}$, a $CNF(\mathcal{S})$-formula over a set of variables $V = \{x_1, \dots, x_n\}$ is a finite
conjunction $C_{1} \land \dots \land C_{m}$, where each $C_i$ is an expression of the form $R(\xi_1,\dots , \xi_k)$, with a $k$-ary
relation $R \in S$, and each $\xi_j$ is a variable from $V$ or one of the constants $0, 1$. A $\textit{solution}$ if a
CNF($\mathcal{S}$)-formula $\varphi$ is an assignment $s = (a_{1}, \dots, a_{n})$ of Boolean values to the variables that makes every
clause of $\varphi$ true. A CNF($\mathcal{S}$)-formula is $\textit{satisfiable}$ if it has at least one solution.

The satisfiability problem SAT($\mathcal{S}$) associated with a finite set $\mathcal{S}$ of logical relations asks:
given a CNF($\mathcal{S}$)-formula $\varphi$, is it satisfiable? All well known restrictions of Boolean satisfiability, such as $3$-SAT,
NOT-ALL-EQUAL $3$-SAT, and POSITIVE $1$-IN-$3$ SAT, can be cast as SAT$(\mathcal{S})$ problems, for a suitable choice of $\mathcal{S}$.

Schaefer identified the complexity of every satisfiability problem SAT($\mathcal{S}$), where $\mathcal{S}$ ranges over all finite sets of
logical relations. To state Schaefer’s main result, we need to define some properties and relations for $R$ and $\mathcal{S}$.

\begin{defn} Let $R$ be a logical relation.
    \begin{enumerate}
        \item $R$ is \textit{bijunctive} if it is the set of solutions of a $2$-CNF formula.
        \item $R$ is \textit{Horn} if it is the set of solutions of a Horn formula, where a Horn formula is a CNF-formula where each $C_i$ has at most
        one positive literal.
        \item $R$ is \textit{dual Horn} if it is the set of solutions of a dual Horn formula, where a dual Horn formula is a CNF-formula where each
        $C_i$ has at most one negative literal.
        \item $R$ is \textit{affine} if it is the set of solutions of a system of linear equations over $\mathcal{Z}_2$.
    \end{enumerate}
\end{defn}

\begin{defn}A set $S$ of logical relations is Schaefer if at least one of the following conditions holds:
    \begin{enumerate}
        \item Every relation in $\mathcal{S}$ is bijunctive.
        \item Every relation in $\mathcal{S}$ is Horn.
        \item Every relation in $\mathcal{S}$ is dual Horn.
        \item Every relation in $\mathcal{S}$ is affine.
    \end{enumerate}
\end{defn}

We are now ready to present Schaefer's theorem :
\begin{theorem}{(Schaefer’s Dichotomy Theorem \cite{schaefer_complexity_1978})} Let $S$ be a finite set of logical relations. If $\mathcal{S}$ is Schaefer, then
SAT($\mathcal{S}$) is in $\P$; otherwise, SAT($\mathcal{S}$) is $\NP$-complete.
\end{theorem}\label{theorem:dichotomy}

Theorem \ref{theorem:dichotomy} is called a dichotomy theorem because Ladner \cite{ladner} has shown that if $\P \neq \NP$, then there are problems
in $\NP$ that are neither in $\P$, nor $\NP$-complete. Thus, Theorem \ref{theorem:dichotomy} asserts that no SAT($\mathcal{S}$) problem
is a problem of the kind discovered by Ladner.



% ----------------------------------------------------------------------------------------------------------------------------------
% ~~~~~~~~~~~~~~~~~~~~~~~~~~~~~~~~~~~~~~~~~~~~~~~~~~~~~~~~~~~~~~~~~~~~~~~~~~~~~~~~~~~~~~~~~~~~~~~~~~~~~~~~~~~~~~~~~~~~~~~~~~~~~~~~~~
% ----------------------------------------------------------------------------------------------------------------------------------
\section{Gopalan et al.'s results}\label{sec:gopalan-et-al.'s-results}

Gopalan et al. continued expanding Schaefer's dichotomie and focused on the connectivity of Boolean satisfiability. They believed
that the connectivity properties of Boolean satisfiability merit some study in their own right, as they shed light on the structure of the solution
space of Boolean formulas thus specifically addressed CNF($\mathcal{S}$)-formulas and studied the complexity of the
following two decision problems,
\begin{itemize}
    \item the connectivity problem $Conn(S)$, that asks for a given $CNF(S)$-formula $\varphi$ whether $G(\varphi)$ is connected,
    \item the st-connectivity problem $st-Conn(S)$, that asks for a given $CNF(S)$-formula $\varphi$ and two solutions $s$ and $t$ whether
    there a path from $s$ to $t$ in $G(\varphi)$.
\end{itemize}

\subsection{Complexity-theoric dichotomies}
Schaefer showed that the satisfiability problem is solvable in polynomial time precisely for formulas built from Boolean
relations all of which are bijunctive, or all of which are Horn, or all of which are dual Horn, or all of which are affine. Gopalan et al.
identified new classes of Boolean relations, called \textit{tight relations}, that properly contain the classes of bijunctive, Horn, dual Horn,
and affine relations. They then showed that st-connectivity is solvable in linear time for formulas built from tight relations,
and $\PSPACE$-complete in all other cases. Their second main result is a dichotomy theorem for the connectivity
problem: it is in $\coNP$ for formulas built from tight relations, and $\PSPACE$-complete in all other cases.

\subsection{Structural dichotomy theorem}
In addition to these two complexity-theoretic dichotomies, they established a structural dichotomy theorem for the diameter of the connected
components of the solution space of Boolean formulas. This result asserts that, in the $\PSPACE$-complete cases, the diameter of
the connected components can be exponential, but in all other cases it is linear. Thus, small
diameter and tractability of the st-connectivity problem are remarkably aligned.

Their results are summarized in comparison to the satisfiability problem SAT($\mathcal{S})$ in table \ref{tab:1}.
\begin{table}[h!]
    \centering
    \begin{tabular}{|c || c | c | c | c |}
        \hline
        $\mathcal{S}$ & SAT($\mathcal{S}$) & ST-CONN($\mathcal{S}$) & CONN($\mathcal{S}$) & Diameter \\ [0.5ex]
        \hline\hline
        Schaefer & $\P$ & $\P$ & $\coNP$ & $O(n)$ \\
        Tight, not Schaefer & $\NP$-compl. & $\P$ & $\coNP$-compl. & $O(n)$ \\
        Not tight & $\NP$-compl. & $\PSPACE$-compl. & $\PSPACE$-compl. & $2^{\Omega (\sqrt{n}) }$ \\ [1ex]
        \hline
    \end{tabular}
    \caption{Gopalan et al.’s results \cite{gopalan_connectivity_2006}}
    \label{tab:1}
\end{table}


\section{A computational Trichotomy for Conn(S)}\label{sec:a-computational-trichotomy}
Schwerdtfeger investigated issues in Gopalan et al.'s work and argued that repeated occurrences of variables in constraint applications
can make the problems harder and the diameter exponential in some cases. This lead to a slight shift of the boundaries established by
Gopalan et al.'s in the hard direction and an introduction to the new classes called \textit{safely tight} and relations \textit{CPSS} fitted to
the correct boundaries.


The following table summarizes their results.
\begin{table}[h!]
    \centering
    \begin{tabular}{|c || c | c | c | c |}
        \hline
        $\mathcal{S}$ & SAT($\mathcal{S}$) & ST-CONN($\mathcal{S}$) & CONN($\mathcal{S}$) & Diameter \\ [0.5ex]
        \hline\hline
        CPSS & $\P$ & $\P$ & $\P$ & $O(n)$ \\
        Schaefer, not CPSS & $\P$ & $\P$ & $\coNP$-compl. & $O(n)$ \\
        Safely tight, not Schaefer & $\NP$-compl. & $\P$ & $\coNP$-compl. & $2^{\Omega (\sqrt{n}) }$ \\
        Not safely tight & $\NP$-compl. & $\PSPACE$-compl. & $\PSPACE$-compl. & $2^{\Omega (\sqrt{n}) }$ \\ [1ex]
        \hline
    \end{tabular}
    \caption{Complete classification of the connectivity problems and the diameter
    for CNF($\mathcal{S}$)-formulas with constants, in comparison to SAT \cite{schwerdtfeger_computational_2015}}
    \label{tab:2}
\end{table}


