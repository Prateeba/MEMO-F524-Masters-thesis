\chapter{Reconfiguration of satisfiability problems} \label{chap:SAT}
In this chapter, boolean satisifiability reconfiguration problems are presented. For decades the Boolean satisfiability problem also known
as SAT has fascinated scientific world. The craze behind it led to the celebrated theorem of Cook Levin, that SAT is $\NP$-complete. In the
reconfiguration context the following general set-up is considered :
Given a Boolean formula $\Phi$ with $n$ Boolean variable and two satisfying assignments $s_{0}$ and $s_{t}$, is it possible to
transform $s_{0}$ to $s_{t}$ such that at each step, only one variable $x_i$ can be flipped and each intermediate assignment remains
feasible.

For Boolean satisfiability problems, the structure of the solution space
is characterized by the solution graph, where the vertices are the solutions,
and two solutions are connected iff they differ in exactly one variable. In
2006, Gopalan et al. studied connectivity properties of the solution graph. motivated mainly by
research on satisfiability algorithms and the satisfiability threshold. They
proved dichotomies for the diameter of connected components and for the
complexity of the st-connectivity question, and conjectured a trichotomy
for the connectivity question. Recently, we were able to establish the
trichotomy \cite{schwerdtfeger2013computational}.

The Boolean satisfiability problem (SAT), as well as many related questions like
equivalence, counting, enumeration, and numerous versions of optimization, are
of great importance in both theory and applications of computer science. In this
article, we focus on the solution-space structure: We consider the solution graph
,
where the vertices are the solutions, and two solutions are connected iff they
differ in exactly one variable. For this implicitly defined graph, we then study
the connectivity and st-connectivity problems, and the diameter of connected
components. The figures below give an impression of how solution graphs may
look like.

A direct application of st-connectivity in solution graphs are reconfiguration
problems, that arise when we wish to find a step-by-step transformation between two feasible solutions of a problem,
such that all intermediate results are also feasible.

Meanwhile, Gopalan et al.’s results have also been applied directly to reconfiguration problems, that arise when a step-by-step transformation
between two feasible solutions of a problem is searched, such that all intermediate results are feasible. The solutions
(satisfying assignments) of a formula $\Phi$ over n variables induce  a subgraph $G(\Phi)$ of the n-dimensional hypercube graph, that is, the
vertices are the solutions of $\Phi$, and two solutions are connected iff they differ in exactly one
variable.
% ------------------------------------------------------------------------------------------------------

The Boolean satisfiability problem (satisfiability problem for short) is one of the central problems in computational
complexity theory. Schaefer proposed in \cite{schaefer_complexity_1978} a framework for expressing variants of the satisfiability problem,
and showed a dichotomy theorem: the satisfiability problem for certain classes of Boolean formulas is in $\P$ while it is $\NP$-complete for
all other classes in the framework. In a single stroke, this result pinpoints the computational
complexity of all well-known variants of SAT, such as $3$-SAT, HORN $3$-SAT, NOT-ALL-EQUAL $3$-SAT, and 1-IN-$3$ SAT.

Since then, dichotomies or trichotomies have been established
for several aspects of the satisfiability problem such as optimization [6,8,24], counting [7], inverse satisfiability [23], minimal
satisfiability [28], unique satisfiability [19], 3-valued satisfiability [3] and propositional abduction [9].
Very recently, Gopalan et al. studied in [17,18] connectivity properties of the solution-space of Boolean formulas, and
investigated complexity issues on connectivity problems in Schaefer’s framework [31], while the connectivity properties
of disjunctive normal forms (DNFs) were studied by Ekin et al. [14]. The connectivity problem (Conn) is to decide whether
the solutions of a given Boolean formula $\Phi$ on n variables induce a connected subgraph of the n-dimensional hypercube,
while the st-connectivity problem (st-Conn) is to decide whether two specific solutions s and t of $\Phi$ are connected. As
mentioned in [17,18], connectivity properties of Boolean satisfiability merit study in their own right, since they shed light

\todo{Finish intro + results + roadmap}

- Introduce the chapter, it's purpose
- Important results concerning the chapter/SAT
- Roadmap.


\section{Schaefer's framework}

\subsection{Basic concepts}
A CNF-formula is a Boolean formula of the form $C_{1} \land \dots \land C_{n}$, where each $C_i$ is a clause, that is, a finite disjunction
of literals. A $k$-CNF formula $(k \geq 1)$ is a CNF-formula where each $C_i$ has at most $k$ literals.

A \textit{logical relation} $R$ is a non-empty subset of $\{0,1\}^k$, for some $k \geq 1;$ $k$ is the \textit{arity} of $R$. A logical relation
is a function that takes as input a Boolean vector and returns a Boolean. For a set $\mathcal{S}$ of logical relations, a $\mathcal{S}$-formula
is a conjuntion of logical relations from $\mathcal{S}$ , where the arguments of each relation are freely chosen among a set of variables.



Let $\mathcal{S}$ be a finite set of logical relations. A CNF($\mathcal{S}$)\textit{-formula} over a set of variables $V = \{x_{1}, \dots, x_{n}\}$
is a finite conjunction $C_{1} \land \dots \land C_{n}$ of clauses built using relations from $\mathcal{S}$, variables from $V$, and the constants $0$ and $1$;
this means that each $C_{i}$ is an expression of the form $R()$, where $R \in \mathcal{S}$ is a relation of arity $k$, and each $C_{i}$ is a
variable in $V$ or one of the constants $0,1$. A $\textit{solution}$ if a CNF($\mathcal{S}$)-formula $\varphi$ is an assignment
$s = (a_{1}, \dots, a_{n})$ of Boolean values to the variables that makes every clause of $\varphi$ true. A CNF($\mathcal{S}$)-formula is
$\textit{satisfiable}$ if it has at least one solution.

\todo{To clean up}

\subsection{Statement of Results}
The $\textit{satisfiability problems}$ SAT($\mathcal{S}$) associated with a finite set $\mathcal{S}$ of logical relation asks :
Given a CNF($\mathcal{S}$) $\varphi$, is it satisfiable ?

\begin{theorem}Let $\mathcal{S}$ be a finite set of logical relations. If $\mathcal{S}$ is Schaefer, then SAT($\mathcal{S}$) is in
$\P$;  otherwise SAT($\mathcal{S}$) is $\NP-$complete.
\end{theorem}


\subsubsection{Dichotomie Results}

\subsection{Connectivity of CNF-Formulas}
Research has focused on the structure of the solution space only quite recently: One of the earliest studies on solution-space connectivity
was done for CNF($\mathcal{S}$)-formulas. In this paper, we are interested in the connectivity properties of the space of solutions
of CNF($\mathcal{S}$)-formulas. If $\varphi$ is a CNF($\mathcal{S}$)-formula with $n$ variables, then the solution graph
G($\varphi$) of $\varphi$ denotes the subgraph of the $n$-dimensional hypercube induced by the solutions of
$\varphi$. This means that the vertices of G($\varphi$) are the solutions of $\varphi$, and there is an edge between
two solutions of G($\varphi$) precisely when they differ in exactly one variable.

The following decision problems were considered in this context :
Gopalan et al. studied the following two decision problems for CNF($\mathcal{S}$)-formulas:

\begin{flushleft}
    The \textit{st-Connectivity Problem} ST-CONN($\mathcal{S}$) \\
    \textbf{Instance: } A CNF($\mathcal{S}$)-formula $\varphi$ and two satisfying assignments $s$ and $t$ of $\Phi$. \\
    \textbf{Question: } Is there a path from $s$ to $t$ in G($\varphi$) ? \\
\end{flushleft}

\begin{flushleft}
    The \textit{Connectivity Problem} CONN($\mathcal{S}$) \\
    \textbf{Instance: } A CNF($\mathcal{S}$)-formula $\Phi$ . \\
    \textbf{Question: } Is G($\Phi$) connected ? \\
\end{flushleft}




\section{Hardness Resulrs for $3$-CNF formulas}







