\chapter{Conclusion}
We have analysed various aspect of Bounded NCL and have shown that
despite some ambiguities, Planar Bounded NCL is indeed NP-complete.
We have reduced Bounded NCL to both Klondike and Mahjong and

verified the prior results that these puzzles are NP-complete. We have re-
duced Planar Bounded NCL to Nonograms, and verified the prior result that
Nonograms are NP-complete. We consider our contribution on these results
amongst the verification, also the simplicity of the established proof. We have
reduced Bounded 2CL to Dou Shou Qi, and proven it PSPACE-hard.

By doing so we have added weight to the claim of [6], stating that Con-
straint Logic is indeed a decent framework to analyse and proof the complex-
ity of puzzles and games.

As a suggestion for future work it would be interesting to see whether
Dou Shou Qi is also in PSPACE, so we can classify it as PSPACE-complete.
If it turns out not to be in PSPACE, a mayor contribution would be to prove
it EXPTIME-hard (for Constraint Logic can provide 2CL) or maybe even
EXPSPACE-hard.

Besides the games and puzzles that are currently classified, there are
games which seem to yield to a reduction from Constraint Logic, for exam-
ple the Binary Puzzle and Rummikub. Reducing the appropriate Constraint
Logic problem to one of these games would be an interesting result.

% ------------------------------------------------------------------------
We have studied the parameterized complexity of constraint logic problems with regards
to (combinations of) solution length, maximum degree and treewidth as parameters. As
a main result, we showed that Nondeterministic Constraint Logic is PSPACE-
complete on planar, 3-regular graphs of bounded bandwidth constructed using only
AND and (protected) OR vertices. We gave several applications of this result, showing
reconfiguration versions of several classical graph problems PSPACE-hard on planar
graphs of bounded bandwidth, as well as showing Rush Hour PSPACE-complete when
played on a rectangular board where one of the lengths of the sides is bounded by a
constant.

We showed that when parameterized by solution length, C2E, C2C and bounded C2E
variants of Nondeterministic Constraint Logic are W[1]-hard, but bounded C2C
NCL becomes fixed parameter tractable. Essentially, the hardness of bounded C2E NCL
when parameterized by solution length is due to the complexity of finding a suitable
target configuration, rather than from that of determining a reconfiguration sequence.
However, when parametrizing by maximum degree in addition to solution length, all
cases become fixed parameter tractable.

When parameterized by treewidth, Constraint Graph Satisfiability becomes weakly
NP-complete (rather than strongly NP-complete). When considering maximum degree
in addition to treewidth, CGS becomes fixed parameter tractable. Note that this is
in stark contrast to the complexity of NCL, which remains PSPACE-complete even for
fixed values of this parameter. This is an interesting example of how reconfiguration
problems can be much harder than their decision variants.

\begin{figure}[H]
    \begin{center}
        \begin{scaletikzpicturetowidth}{\textwidth}
            \begin{tikzpicture}[scale=1]
                \node (1) at (0,0) [draw, rounded rectangle, align=center] {CONFIGURATION-TO-EDGE\\for restricted NCL};
                \node (2)  [below=of 1, draw, rounded rectangle, align=center] {SLIDING-TOKEN\\problem};
                \node (3)  [below=of 2, draw, rounded rectangle, align=center] {STANDARD SLIDING-TOKEN\\problem};
                \node (4)  [below=of 3, draw, rounded rectangle, align=center] {LIST-COLOUR PATH\\problem};
                \node (5)  [below=of 4, draw, rounded rectangle, align=center] {$K$-COLOUR PATH\\problem};
                \node (6) at (8, -2.3) [draw, rounded rectangle, align=center] {Labelled sliding token\\problem};
                \node (7) [below=of 6, draw, rounded rectangle, align=center] {Exact cover split and merge\\ reconfiguration problem};
                \node (8) [below=of 7, draw, rounded rectangle, align=center] {$k$-move Subset sum\\reconfiguration problem};

                \draw[->]  (1) to node [auto] {$\leq{p}$} (2);
                \draw[-]  (2) to node [auto] {variant} (3);
                \draw[->]  (3) to node [auto] {$\leq{p}$} (4);
                \draw[->]  (4) to node [auto] {$\leq{p}$} (5);
                \draw[-]  (2) to node [auto] {variant} (6);
                \draw[->]  (6) to node [auto] {$\leq{p}$} (7);
                \draw[->]  (7) to node [auto] {$\leq{p}$} (8);

            \end{tikzpicture}
        \end{scaletikzpicturetowidth}
    \end{center}
    \caption{$\PSPACE$-complete problems encoutered and their relationship.}\label{fig:conclusion}
\end{figure}