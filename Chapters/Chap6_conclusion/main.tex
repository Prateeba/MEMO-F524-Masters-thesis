\chapter{Conclusions and Future Work}
Throughout this thesis, we have researched and analysed reconfiguration problems by starting with the Nondeterministic Constraint Logic
which seems to be the go-to reduction in order to prove $\PSPACE$-hardness results. More specifically, we focused on the alternative
formulation of NCL and detailed the $\PSPACE$-completeness result of the sliding token problem. We then showed that the labelled variant of the
SLIDING TOKEN problem is also $\PSPACE$-complete. This latter result was then used to establish the complexity result of the
$k$-move Subset sum reconfiguration problem for $k = 3$. Our contribution was to provide detailed explanatory examples and explications to complement
the proof of theorem \ref{theorem:3_move_theorem}.
All reconfiguration problems encountered during our journey to the completion of this thesis is summed in fig \ref{fig:conclusion}.

\begin{figure}[H]
    \begin{center}
        \begin{scaletikzpicturetowidth}{\textwidth}
            \begin{tikzpicture}[scale=0.8]
                \node (1) at (0,0) [draw, rounded rectangle, align=center] {CONFIGURATION-TO-EDGE\\for restricted NCL};
                \node (2)  [below=of 1, draw, rounded rectangle, align=center] {SLIDING-TOKEN\\problem};
                \node (3)  [below=of 2, draw, rounded rectangle, align=center] {STANDARD SLIDING-TOKEN\\problem};
                \node (4)  [below=of 3, draw, rounded rectangle, align=center] {LIST-COLOUR PATH\\problem};
                \node (5)  [below=of 4, draw, rounded rectangle, align=center] {$K$-COLOUR PATH\\problem};
                \node (6) at (8, -2.3) [draw, rounded rectangle, align=center] {Labelled sliding token\\problem};
                \node (7) [below=of 6, draw, rounded rectangle, align=center] {Exact cover split and merge\\ reconfiguration problem};
                \node (8) [below=of 7, draw, rounded rectangle, align=center] {$k$-move Subset sum\\reconfiguration problem};

                \draw[->]  (1) to node [auto] {$\leq{p}$} (2);
                \draw[-]  (2) to node [auto] {variant} (3);
                \draw[->]  (3) to node [auto] {$\leq{p}$} (4);
                \draw[->]  (4) to node [auto] {$\leq{p}$} (5);
                \draw[-]  (2) to node [auto] {variant} (6);
                \draw[->]  (6) to node [auto] {$\leq{p}$} (7);
                \draw[->]  (7) to node [auto] {$\leq{p}$} (8);

            \end{tikzpicture}
        \end{scaletikzpicturetowidth}
    \end{center}
    \caption{$\PSPACE$-complete problems encountered and their relationship.}\label{fig:conclusion}
\end{figure}